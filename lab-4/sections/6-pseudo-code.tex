\subsection*{Задание 1}

Константа MaxSize = 100\\

\noindent \underline{Обьявить переменные:}\\
\null\qquad rows, cols, sizeOfArray: целое число\\
\null\qquad matrix: массив размером MaxSize * MaxSize\\
\null\qquad array: массив размером MaxSize\\

\noindent \underline{функция} find(element: целое число):\\
\null\qquad \underline{цикл} i := 0 \underline{до} sizeOfArray - 1:\\
\null\qquad \qquad \underline{если} array[i] = element \underline{то}\\
\null\qquad \qquad \qquad \underline{вернуть} true\\
\null\qquad \qquad \underline{всё}\\
\null\qquad \underline{кц}\\
\null\qquad \underline{вернуть} false\\
\underline{кц}\\

\noindent \underline{процедура} inputArray(size: целое число, array: массив целых чисел):\\
\null\qquad correctInput: булево\\
\null\qquad \underline{цикл}:\\
\null\qquad \qquad correctInput = true\\
\null\qquad \qquad \underline{цикл} i := 0 \underline{до} size - 1:\\
\null\qquad \qquad \qquad ввод(array[i])\\
\null\qquad \qquad \qquad \underline{если} array[i] не является целым числом \underline{то}\\
\null\qquad \qquad \qquad \qquad correctInput = false\\
\null\qquad \qquad \qquad \underline{всё}\\
\null\qquad \qquad \underline{кц}\\
\null\qquad \qquad \underline{если} !correctInput \underline{то}\\
\null\qquad \qquad \qquad вывод("Неверный формат ввода. Попробуйте снова.")\\
\null\qquad \qquad \underline{всё}\\
\null\qquad \underline{до} correctInput\\
\null\qquad \underline{кц}\\
\underline{кц}\\

\noindent \underline{функция} validSize(size: целое число):\\
\null\qquad \underline{вернуть} size > 0 и size <= MaxSize\\
\underline{кц}\\

\noindent \underline{Нач}\\
\null\qquad вывод("Введите размеры матрицы в следующем формате:")\\
\null\qquad вывод("'число строк' 'число столбцов'")\\
\null\qquad вывод("Оба числа должны быть положительными целыми")\\

\noindent
\null\qquad \underline{цикл-пока} ввод(rows) не целое или ввод(cols) не целое или !validSize(rows) или !validSize(cols):\\
\null\qquad \qquad вывод("Неверный формат ввода. Попробуйте снова.")\\
\null\qquad \underline{кц}\\

\noindent
\null\qquad вывод("Введите матрицу в следующем формате:")\\
\null\qquad вывод("matrix[0][0] matrix[0][1] ... matrix[1][1] ... matrix[rows][cols]")\\
\null\qquad inputArray(rows * cols, matrix)\\

\noindent
\null\qquad вывод("Введите размер массива (положительное целое число).")\\

\noindent
\null\qquad \underline{цикл-пока} ввод(sizeOfArray) не целое или !validSize(sizeOfArray):\\
\null\qquad \qquad вывод("Неверный формат ввода. Попробуйте снова.")\\
\null\qquad \underline{кц}\\

\noindent
\null\qquad вывод("Введите массив в следующем формате:")\\
\null\qquad вывод("array[0] array[1] ... array[sizeOfArray]")\\
\null\qquad inputArray(sizeOfArray, array)\\

\noindent
\null\qquad \underline{цикл} i := 0 \underline{до} rows - 1:\\
\null\qquad \qquad maximumIndex = 0\\
\null\qquad \qquad \underline{цикл} j := 0 \underline{до} cols - 1:\\
\null\qquad \qquad \qquad \underline{если} abs((matrix[i * cols + j])) > abs((matrix[i * cols + maximumIndex])) \underline{то}\\
\null\qquad \qquad \qquad \qquad maximumIndex = j\\
\null\qquad \qquad \qquad \underline{всё}\\
\null\qquad \qquad \underline{кц}\\
\null\qquad \qquad \underline{цикл} j := 0 \underline{до} cols - 1:\\
\null\qquad \qquad \qquad \underline{если} abs((matrix[i * cols + j])) = abs((matrix[i * cols + maximumIndex])) \underline{то}\\
\null\qquad \qquad \qquad \qquad matrix[i * cols + j] = 0\\
\null\qquad \qquad \qquad \underline{всё}\\
\null\qquad \qquad \underline{кц}\\
\null\qquad \underline{кц}\\

\noindent
\null\qquad вывод("Результат:")\\
\null\qquad \underline{цикл} i := 0 \underline{до} rows - 1:\\
\null\qquad \qquad \underline{цикл} j := 0 \underline{до} cols - 1:\\
\null\qquad \qquad \qquad вывод(matrix[i * cols + j])\\
\null\qquad \qquad \underline{кц}\\
\null\qquad \underline{кц}\\

\noindent
\noindent \underline{Кон}\\

\subsection*{Задание 2}

Константа MaxSize = 100\\

\noindent
\underline{Обьявить переменные}:\\
\null\qquad sizeOfArray: целое число\\
\null\qquad array: массив размером MaxSize\\

\noindent
\underline{функция} sumOfDigits(number: целое число):\\
\null\qquad sum = 0\\
\null\qquad \underline{цикл-пока} number > 0:\\
\null\qquad \qquad sum += number \% 10\\
\null\qquad \qquad number /= 10\\
\null\qquad \underline{кц}\\
\null\qquad \underline{вернуть} sum\\
\underline{кц}\\

\noindent \underline{процедура} inputArray(size: целое число, array: массив целых чисел):\\
\null\qquad correctInput: булево\\
\null\qquad \underline{цикл}:\\
\null\qquad \qquad correctInput = true\\
\null\qquad \qquad \underline{цикл} i := 0 \underline{до} size - 1:\\
\null\qquad \qquad \qquad ввод(array[i])\\
\null\qquad \qquad \qquad \underline{если} array[i] не является целым числом \underline{то}\\
\null\qquad \qquad \qquad \qquad correctInput = false\\
\null\qquad \qquad \qquad \underline{всё}\\
\null\qquad \qquad \underline{кц}\\
\null\qquad \qquad \underline{если} !correctInput \underline{то}\\
\null\qquad \qquad \qquad вывод("Неверный формат ввода. Попробуйте снова.")\\
\null\qquad \qquad \underline{всё}\\
\null\qquad \underline{до} correctInput\\
\null\qquad \underline{кц}\\
\underline{кц}\\

\noindent
\underline{функция} validSize(size: целое число):\\
\null\qquad \underline{вернуть} size > 0 и size <= MaxSize\\
\underline{кц}\\

\noindent
\underline{Нач}:\\
\null\qquad вывод("Введите размер массива (положительное целое число).")\\

\noindent
\null\qquad \underline{цикл-пока} ввод(sizeOfArray) не целое или !validSize(sizeOfArray):\\
\null\qquad \qquad вывод("Неверный формат ввода. Попробуйте снова.")\\
\null\qquad \underline{кц}\\

\noindent
\null\qquad вывод("Введите элементы массива (целые числа) в следующем формате:")\\
\null\qquad вывод("array[0] ... array[sizeOfArray]")\\
\null\qquad inputArray(sizeOfArray, array)\\

\noindent
\null\qquad \underline{цикл} i := 0 \underline{до} sizeOfArray - 1:\\
\null\qquad \qquad array[i] = sumOfDigits(array[i])\\
\null\qquad \underline{кц}\\

\noindent
\null\qquad вывод("Результат:")\\
\null\qquad \underline{цикл} i := 0 \underline{до} sizeOfArray - 1:\\
\null\qquad \qquad вывод(array[i])\\
\null\qquad \underline{кц}\\

\noindent
\underline{Кон}\\
